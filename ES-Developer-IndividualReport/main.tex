\documentclass[12pt, a4paper]{article}
\usepackage[utf8]{inputenc}
% \usepackage{pdflscape}
\usepackage{rotating}
% \usepackage{lscape}
\usepackage{amssymb}
\usepackage{indentfirst}
\usepackage{listings}
\usepackage{enumitem}
\usepackage{comment}
\usepackage{graphicx}
\usepackage{color}
\usepackage[portuguese]{babel}
\usepackage{geometry}
\geometry{legalpaper, a4paper,
 total={170mm,257mm},
 left=20mm,
 top=20mm}
\setlength{\voffset}{-10mm}
\definecolor{dkgreen}{rgb}{0,0.6,0}
\definecolor{gray}{rgb}{0.5,0.5,0.5}
\definecolor{mauve}{rgb}{0.58,0,0.82}

\lstset{frame=tb,
  language=Java,
  aboveskip=3mm,
  belowskip=3mm,
  showstringspaces=false,
  columns=flexible,
  basicstyle={\small\ttfamily},
  numbers=none,
  numberstyle=\tiny\color{gray},
  keywordstyle=\color{blue},
  commentstyle=\color{dkgreen},
  stringstyle=\color{mauve},
  breaklines=true,
  breakatwhitespace=true,
  tabsize=3
}

\usepackage{hyperref}
\hypersetup{
    colorlinks=true,
    linkcolor=blue,
    filecolor=magenta,      
    urlcolor=cyan,
}


\newcommand{\tit}[1]{\textit{#1}}
\newcommand{\tb}[1]{\textbf{#1}}
\newcommand{\tbi}[1]{\textbf{\textit{#1}}}

\newcommand{\bitem}[2]{ \tb{(\tit{#1}) {#2}}}
\newcommand{\iitem}[1]{(\tit{#1})}

\newcommand{\oo}{orientação à objetos}
\newcommand{\sw}{software}
\newcommand{\ssw}{software }
\newcommand{\vv}{V\&V}

\newcommand{\question}[1]{\item {#1}}
\newcommand{\answer}[1]{\par \tb{Resposta:} #1}

\newcommand{\quotes}[1]{``#1''}

\title{INF01127 - Relatório Individual}

\author{Wellington M. Espindula}
\date{Maio de 2021}

\begin{document}
    \maketitle
    
    \section{Product Owner}
    Em meio a muitas reuniões e discussões, posso afirmar que a etapa de Product Owner, para mim - que me identifico mais com um perfil técnico -, sem dúvidas foi a parte mais desafiadora. Nessa etapa, tivemos que expôr ideias, lidar com antíteses, ser criativos e pensar sobretudo no cliente e no que gostaríamos que o nosso produto fosse capaz de resolver para estes. Para tanto, das vastas possibilidades que a \tit{Internet of Things} oferece, surgiram várias ideias no grupo, desde aplicações de segurança até \tit{Internet of Medical Things.} Sendo assim, pensamos em criar uma solução que pudesse lidar com monitores pré-existentes em um hospital, tornado-os inteligentes e interconectando-os através de IoT a fim de criarmos alertas redirecionados aos médicos com o propósito de diminuir o tempo de resposta da equipe médica à emergências. 
    
    A partir do momento que tivemos essa ideia e quisemos segui-la, tivemos que destrinchá-la para pensarmos em como a venderíamos para um cliente e como seria a sua experiência com o produto. Assim tivemos que pensar em quais seriam as condições de uso, como o usuário gostaria de usá-lo e quais \tit{features} iríamos ou não adicionar. E talvez isso tenha sido tão difícil quanto pensar na ideia do produto em si. Mas como sempre lidamos com esses problemas em grupo, pensando e discutindo sempre de forma coletiva, o peso da tarefa se dividiu e pudemos idealizar o produto e descrevê-lo, possibilitanto dar prosseguimento ao desenvolvimento - com uma visão um pouco mais clara do almejávamos.
    
    \section{Developer}
    Para a etapa de desenvolvimento, o maior desafio foi o processo de planejamento da arquitetura do software. Para tal, envolvemos diversas reuniões em grupo a fim de desenvolvermos uma arquitetura limpa, com módulos com baixo acoplamento e altamente coesos. Inicialmente, imaginamos uma arquitetura totalmente centralizada através de um \ssw central que iria ficar monitorando e realizando o controle de tudo. Após o \tit{feedback} da etapa 2, resolvemos realizar uma mudança arquitetural brusca buscando uma maior descentralização dos componentes entre si e comunicação direta entre os dispositivos. Entretanto, ainda foi necessário um componente intermediário chamado \tit{DataCenter} para a persistência de dados (como \tit{Logs} dos dispositivos e registros de pacientes). 
    
    Apesar disso, o processo de desenvolvimento não foi tão linear quanto possa aparecer. A arquitetura foi se modificando ao longo do processo de desenvolvimento a fim de gerar códigos limpos e concisos com uma arquitetura consistente e módulos muito bem distribuídos. Em relação ao processo de desenvolvimento, inicialmente, coletivamente, definimos as classes bases do sistema (componentes principais, interfaces e classes abstratas). A partir de então, o trabalho foi mais distribuído produzindo as implementações e a orquestração. Em particular, eu fui responsável pelo desenvolvimento da interface do app (app de um médico) - muito embora meus colegas tenham sido altamente prestativos e tenham realizados muitas contribuições significativas no mesmo -, e atuei ativamente na orquestração dos componentes e em demais tarefas do desenvolvimento.
    
    Por fim, posso afirmar que o trabalho me agregou muito em como eu enxergo o desenvolvimento de \sw, aumentando a minha visão panorâmica sobre o desenvolvimento desde a sua concepção. Acredito que ter um grupo unido discutindo todas as etapas do processo colaborou muito para isto. Dessa forma, o trabalho em grupo possibilitou pormos em prática o que vimos ao longo do semestre, agregando tanto o conhecimento teórico e prático de forma a consolidarmos o conhecimento da disciplina.
    

    
\end{document}